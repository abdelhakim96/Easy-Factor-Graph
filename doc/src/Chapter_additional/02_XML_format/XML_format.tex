\newpage
\chapter{Import models from xml files}
\label{00_XML_format}

The aim of this Section is to expose how to build graphical models from XML files describing their structures.
In particular, the syntax of such an XML will be clarified.
Figure \ref{fig:00:XML_struct} visually explains the structure of a valid XML.
\\
Essentially two kind of tags must be incorporated:
\begin{itemize}
\item Variable: describes the information related to a variable present in the graph. There must a tag of this kind for every variable constituting the model. Fields description:
\begin{itemize}
\item name: is a string indicating the name of this variable.
\item Size: is the size of the variable, i.e. the size of $Dom$, see Section \ref{sec:00:PREL}.
\item flag[optional] : is a flag that can assume two possible values, 'O' or 'H' according to the fact that this variable is in set $\mathcal{O}$ (Section \ref{sec:00:PREL}) or not respectively. When non specifying this flag 'H' is assumed.
\end{itemize}
\item Potential: describes the information related to a unary or a binary potential present in the graph (see Section \ref{sec:00:PREL}).
Fields description:
\begin{itemize}
\item var: the name of the first variable involved.
\item var[optional]: the name of the second variable involved. Is omitted when considering unary potentials, while is mandatory when a binary potentials is described by this tag.
\item weight[optional]: when specifying an Exponential Factors (Section \ref{sec:00:PREL}) it must be present for indicating the value of the weight $w$ (equation (\ref{eq:00:exp_w})). When omitting, the potential is assumed to be a simple Factor.
\item tunability[optional]: it is a flag for specifying whether the weight of this Exponential Factor is tunable or not (see Section \ref{sec:00:PREL}). Is ignored in case weight is omitted. It can assumes two possible values, 'Y' or 'N' according to the fact that the weight involved is tunable or not respectively. When weight is specified and tunability is omitted, a value equal to 'Y' is assumed.
\end{itemize}
\item Share[optional]: you must specify this sub tag when the containing Exponential Factor shares its weight with another potential in the model. 
Sub fields var are exploited for specifying the variables involved by the potential whose weight is to share. If weight is omitted in the containing Potential tag, this sub tag is ignored, even though the value assigned to weight is ignored since it is shared with another potential.
The potential sharing its weight must be clearly an Exponential shape, otherwise the sharing directive is ignored.
\\
\\
The following components are exclusive: only one of them can be specified in a Potential tag and at the same time at least one must be present.
\begin{itemize}
\item Correlation: it can assume two possible values, 'T' or 'F'. When 'T' is passed, this potential is assumed to be a correlating potential (see sample \ref{samples:01:corr_anti}), otherwise when passing 'F' a simple anti correlating factor is assumed.
It is invalid in case this Potential is a unary one.
In case weight was specified, an Exponential Factor is built, passing as input the correlating or anti-correlating Factor described by this tag.
\item Source: it is the location of a textual file describing the values of the distribution characterizing this potential.
Rows of this file contain the values charactering the image of the potential. 
Combinations not specified are assumed to have an image value equal to 0.
Clearly the number of values charactering the distribution must be consistent with the number of specified var fields.
In case weight was specified, an Exponential Factor is built, starting from the Factor described by the values specified in the aforementioned file.
For instance, the potential $\Phi _b$ of Section \ref{sec:00:PREL} would have been described by a file containing the following rows:
\begin{eqnarray}
0 \,\,\, 0 \,\,\, 1 \nonumber\\
0 \,\,\, 1 \,\,\, 4 \nonumber\\
1  \,\,\,1  \,\,\,1 \nonumber\\
2  \,\,\,2  \,\,\,5 \nonumber\\
2  \,\,\,4  \,\,\,1 \nonumber\\
\label{eq:00:XML_struct:shape_txt}
\end{eqnarray}
\item Set of sub tags Distr$\_$val: is a set of nested tags describing the distribution of the this potential.
Similarly to Source, every element use fields v for describing the combination, while D is used for specifying the value assumed by the distribution.
For example the potential $\Phi _b$ of Section \ref{sec:00:PREL} would have been described by the syntax reported in Figure \ref{fig:00:XML_struct:Distr_val}.
In case weight was specified, an Exponential Factor is built, starting from the Factor whose distribution is specified by the aforementioned sub tags.
\end{itemize}
\end{itemize}

\begin{figure}
	\centering
\def\svgwidth{0.95 \columnwidth}
\import{../src/Chapter_additional/02_XML_format/image/}{XML_structure.pdf_tex} 
	\caption{The structure of the XML describing a graphical model.}
	\label{fig:00:XML_struct}
\end{figure} 

\begin{figure}
	\centering
\def\svgwidth{0.6 \columnwidth}
\import{../src/Chapter_additional/02_XML_format/image/}{Distr_val_example.pdf_tex} 
	\caption{Syntax to adopt for describing the potential $\Phi _b$ of Section \ref{sec:00:PREL}, using a population of Distr$\_$val sub tags.}
	\label{fig:00:XML_struct:Distr_val}
\end{figure} 
